% Acknowledgments

\chapter*{Acknowledgments}
\addcontentsline{toc}{chapter}{Acknowledgments}

Writing a comprehensive technical book is never a solitary endeavor. This work stands on the shoulders of countless individuals and communities who have contributed to the Wayland ecosystem and to my understanding of it.

\section*{The Wayland Community}

First and foremost, I must thank the Wayland developers and community members who created this remarkable technology. Special recognition goes to:

\textbf{Kristian Høgsberg}, who initiated the Wayland project in 2008 and established its foundational design principles. Your vision of a simpler, more secure display server has fundamentally changed how we think about graphics on Linux.

\textbf{Daniel Stone}, whose extensive work on Wayland, weston, and related infrastructure has been instrumental in making Wayland practical and performant. Your technical insights and detailed explanations have helped countless developers understand the system.

\textbf{Pekka Paalanen}, for your meticulous work on the Wayland protocol, weston, and your invaluable contributions to documentation and community support. Your patience in answering questions and explaining complex concepts has benefited the entire community.

\textbf{Drew DeVault}, for creating wlroots and demonstrating that building custom Wayland compositors doesn't require reimplementing the entire graphics stack. Your work has democratized compositor development and inspired a new generation of window managers.

\section*{The wlroots Team}

The wlroots project deserves special recognition for making Wayland compositor development accessible:

\textbf{Simon Ser}, for your continued leadership and technical contributions to wlroots, maintaining high standards of code quality and API design.

The entire wlroots community, whose collective work on Sway, river, wayfire, and numerous other compositors has proven the value of this architectural approach.

\section*{Major Compositor Projects}

This book draws heavily on studying real-world implementations. Thanks to the developers of:

\textbf{GNOME Mutter} — Jonas Ådahl, Carlos Garnacho, and the entire GNOME team for pioneering Wayland integration in a full desktop environment.

\textbf{KDE KWin} — Martin Flöser, Vlad Zahorodnii, and the KDE team for their thorough and well-documented Wayland implementation.

\textbf{Sway} — Drew DeVault and the Sway contributors for proving that tiling window managers have a bright future on Wayland.

\textbf{weston} — The reference compositor team for providing an invaluable learning resource and test bed for new protocol features.

\section*{The X11 Legacy}

Understanding Wayland requires understanding what came before. Thanks to the generations of X Window System developers whose work provided both a foundation to build upon and lessons to learn from. The X.Org Foundation's decades of open development have made modern Linux graphics possible.

\section*{Graphics Stack Contributors}

Wayland doesn't exist in isolation. Thanks to:

The \textbf{Mesa} developers for their extraordinary work on open source graphics drivers.

The \textbf{kernel DRM/KMS subsystem} maintainers for providing the foundation upon which modern display servers are built.

The \textbf{libinput} team for creating a robust, compositor-agnostic input handling library.

\section*{Documentation and Education}

Several resources were invaluable in creating this book:

\textbf{The Wayland Book} by Drew DeVault — an excellent practical introduction that inspired me to tackle the broader architectural perspective.

\textbf{The Wayland protocol documentation} — meticulous specification work by the protocol developers.

\textbf{freedesktop.org} and the various wikis and blogs that have documented Wayland over the years.

\section*{Technical Reviewers}

While this is a personal work, I am grateful to the many community members who have answered questions, clarified technical details, and patiently explained concepts in IRC channels, mailing lists, and forums over the years.

\section*{The Open Source Community}

More broadly, thanks to the entire open source community. The culture of sharing knowledge, code, and ideas makes projects like this possible. Every bug report, patch, documentation improvement, and thoughtful discussion contributes to our collective understanding.

\section*{Tools and Infrastructure}

This book was written using entirely open source tools:

\begin{itemize}
    \item \textbf{\LaTeX} for typesetting
    \item \textbf{TikZ} for diagrams
    \item \textbf{Git} for version control
    \item \textbf{Linux} as the development platform
    \item Various text editors and development tools from the FOSS ecosystem
\end{itemize}

The ability to write a technical book using free and open tools, about free and open technology, distributed under a free and open license, is itself a testament to what the open source movement has achieved.

\section*{Personal Thanks}

To my family and friends who tolerated countless hours of "I need to finish this chapter" and understood when I disappeared into the depths of graphics stack documentation.

To the countless developers I've never met whose code I've studied, whose commits I've read, and whose design decisions I've analyzed. Your work is the primary source material for this book.

\section*{Standing on Shoulders}

Isaac Newton famously wrote, "If I have seen further, it is by standing on the shoulders of giants." This book is only possible because of the giants who built Wayland, documented it, debugged it, and refined it over many years.

Any insights in this book are theirs. Any errors are mine alone.

\section*{A Living Acknowledgment}

The Wayland ecosystem continues to evolve. New contributors join every day. Protocols are extended. Compositors are improved. By the time you read this, many more people will have made important contributions.

To everyone who contributes to making Linux graphics better: thank you. This book is a small attempt to document and honor your work.

\vspace{2cm}

\begin{center}
\textit{With gratitude and respect,}\\
\vspace{0.5cm}
\textit{1ay1}\\
\textit{\url{https://github.com/1ay1/wayland_book}}
\end{center}

\clearpage
